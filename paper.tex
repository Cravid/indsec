\documentclass[conference]{IEEEtran}
\usepackage[utf8]{inputenc}
\usepackage[german]{babel}

\usepackage{hyperref}

\usepackage{graphicx}
\graphicspath{figures/}

\makeatletter
\let\@copyrightspace\relax
\makeatother

\begin{document}

\title{MAC Authentication Bypass (MAB)\\ in Industrie 4.0}
\author{
	Umut-Vural Mitiler\\
	u.mitiler@stud.hs-wismar.de
	\and
	Fakultät für Ingenieurwissenschaften\\
	Hochschule Wismar\\
	Master IT-Sicherheit und Forensik\\
	Industrial Security\\
	Gruppe FFM-08
	\and
	David Schunke\\
	d.schunke@stud.hs-wismar.de
}

\maketitle

\thispagestyle{plain}
\pagestyle{plain}

%

\begin{abstract}
In dieser Abhandlung wird die port-based Network Access Control Methode MAC Authentication Bypass (MAB) sowie deren Bedeutung für die Industrie 4.0 diskutiert. Hierzu wird zunächst die grundsätzliche Motivation für Netzwerkzugangskontrolle dargestellt sowie eine Übersicht gegeben, was Industrie 4.0 bedeutet und definiert. Anschließend werden die technischen Grundlagen von MAB sowie den damit verbundenen Mechanismen und Standards - im speziellen Extensible Authentication Protocol (EAP) und IEEE802.1X - herausgearbeitet. Es wird zudem ein typisches Einsatzszenario als Fallback für RADIUS erläutert. Danach wird auf die Relevanz für die Industrie 4.0 sowie den bereits heute existierenden Einsatz in der Praxis eingegangen. Zum Schluss werden die Vorteile bzw. Nutzen sowie Nachteile und Gefahren diskutiert, und mögliche Gegenmaßnahmen kurz dargestellt. 
\end{abstract}

\vspace{1em}

\begin{IEEEkeywords}
MAC Authentication Bypass, MAB, EAP, Extensible Authentication Protocol, IEEE802.1X, dot1x, Network Access Control, NAC, Industrie 4.0, Schutzziele
\end{IEEEkeywords}

%

\vspace{1em}
\section{Motivation}
Die fortschreitende Vernetzung von IT Systemen bahnt sich ihren weg durch alle Bereiche unseres Lebens. Während ständige Vernetzung im Privaten- sowie Dienstleistungsumfeld heute alltäglich für uns sind, ist in der Industrie ein solcher Grad an Vernetzung noch nicht angekommen. So sind häufig im industriellen Umfeld noch Maschinen oder Systeme eingesetzt, welche aufgrund ihres Alters gar keine Vernetzung und Kommunikation erlauben, als auch existierende Vernetzungen lediglich für eine einfachste Form der Kommunikation, wie z.B. Systemüberwachung, nutzen.\\

Industrie 4.0 stellt die vierte und aktuellste Stufe der industriellen Revolution dar. Nachdem die erste industrielle Revolution durch die Mechanisierung von Arbeitsprozessen, die zweite industrielle Revolution durch den Einsatz technischer Hilfsmittel zur Massenproduktion, und die dritte industrielle Revolution erste digitale Hilfsmittel zur Automatisierung geprägt waren, werden in der vierten industriellen Revolution alle Komponenten digitalisiert, vollumfänglich vernetzt sowie mit Intelligenz versehen, sodass diese Komponenten teils voll-autonom dezentrale Entscheidungen treffen und untereinander kommunizieren können. Der Mensch interagiert nur noch eingeschränkt und teilweise über neuartige Schnittstellen (human-computer-interfaces, HCI) mit den Systemen, um diese z.B. bei Konflikten oder benötigten Entscheidungen anzuleiten. Solche Systeme werden Cyberphysische Systeme (CPS) genannt und bilden das Rückgrat der Industrie 4.0. \cite{botthof2015zukunft}\\

Mit Vernetzung und Kommunikation werden gleichzeitig solche Systeme allerdings auch stärker angreifbar und bedingen weitergehender Schutzmechanismen. Während in Office- oder User-Netzwerken Technologien und Protokolle zur Absicherung der Kommunikation bereits lange zum Standard gehören, sind die in der Industrie noch häufig eingesetzten Systeme und Protokolle auf eine Absicherung noch nicht ausgelegt. Gefahren bilden hierbei u.A.

\vspace{.5em}

\renewcommand{\labelenumi}{\alph{enumi})}
\begin{enumerate}
	\item Industriespionage durch Abfangen von Kommunikation oder eindringen in Systeme (Angriff auf die Vertraulichkeit)
	\item Manipulation von Kommunikation, z.B. Steuerbefehlen (Angriff auf die Integrität)
	\item Störung der (Echtzeit-)Kommunikation bis hin zu Denial-of-Service (DoS), sodass ein System komplett ausfällt (Angriff auf die Verfügbarkeit)
	\item Übernahme (Hijacking) von Systemen und missbräuchliche Verwendung oder Erpressung
\end{enumerate}

\vspace{.5em}

Die verschiedenen Gefahren stellen für die unterschiedlichen Zweige der Industrie auch unterschiedliche Schweregrade dar, sind jedoch in der Industrie 4.0 für alle Bereiche ein hochwichtiges Thema.\\

Ein zentraler Teil zur Absicherung der Schutzziele besteht darin, wer überhaupt Zugang zu einem Netzwerk bekommt, analog zur physischen Zugangsabsicherung zu einem Gebäude oder Areal. Im Alltag findet man heute zugangsoffene Wireless LAN nur noch selten. Meistens wird zur Authentifizierung ein Passwort oder auch Pre-shared Key (PSK) benötigt. Jeder mit diesem PSK kann sich zum Netzwerk verbinden. Im Bereich des wired LAN findet man allerdings auch heute noch oft völlig unbeschränkte Netzwerke, sodass jeder mit physischem Zugang sich verbinden könnte.\\

Die PSK Methode ist simpel, hat jedoch insbesondere zwei große Nachteile:
(a) es ist keine Unterscheidung der Teilnehmer möglich, d.h. jeder Teilnehmer hat auch die gleichen Möglichkeiten im Netzwerk, eine Limitierung Einzelner oder nach Gruppen ist nicht möglich, und
(b) je größer die Anzahl an Teilnehmern im Netzwerk, desto schwieriger wird es den PSK auszutauschen und zu erneuern, sodass dies mit der Zeit ein Sicherheitsrisiko darstellen kann.\\

Zur Lösung dieses Problems wurden bereits für Enterprise Netzwerke verschiedene Strategien und Technologien entwickelt, um eine Netzwerkzugangskontrolle (Network Access Acontrol, NAC) teilnehmergenau zu realisieren, und so eine genauere Authentifizierung sowie Autorisierung auf verschiedene Ressourcen in Form von Rechtevergabe zu ermöglichen.\\

Aufgrund des Alters vieler Systeme in der Industrie können diese Technologien allerdings nicht immer einfach übertragen werden, ohne Probleme in der Kompatibilität hervorzurufen, können gleichzeitig aber auch oft nicht einfach durch neue Systeme ausgetauscht werden. Dies betrifft u.A. Netzwerkdrucker, Sensoren, Kameras, ältere Server, usw.\\

Aus diesem Grund werden für solche Systeme Fallback-Lösungen benötigt, sodass diese Systeme sich gegenüber der Netzwerkzugangskontrollinstanz authentifizieren und am Netzwerk teilnehmen können. Eine solche Fallback-Lösung stellt das \emph{MAC Authentication Bypass (MAB)} dar, welches im Folgenden nähere Erläutert wird.

%

\vspace{1em}
\section{Technische Grundlagen}
Zur Realisierung der Netzwerkzugangskontrolle wird eine zentrale Authentifizierungsstelle (Authentication Server) eingesetzt, sogenannte AAA-Server (Authentication, Authorization, Accounting). Jede Netzwerkkomponente (Authenticator), z.B. Router oder Switch, vergewissert sich bei ankommendem Netzwerkverkehr beim Authentication Server darüber, ob der Sender (Supplicant) authorisiert ist am Netzwerk teilzunehmen. Das \emph{Institute of Electrical and Electronics Engineers (IEEE)} hat hierfür den Standard \emph{IEEE802.1X} \cite{5409813} entwickelt, welcher wiederum die Nutzung des \emph{Extensible Authentication Protocol (EAP)} empfiehlt und somit quasi eine Verkapselung des EAP in einem Standard darstellt, weshalb im Folgenden zunächst das EAP erläutert wird.

\vspace{.5em}
\subsection{Extensible Authentication Protocol (EAP)}
Das EAP (RFC 3748 \cite{aboba2004extensible}) ist ein von der \emph{Internet Engineering Task Force (IETF)} entwickeltes Framework zur Implementierung von Authentifizierung in Netzwerken. Es stellt Funktionen für verschiedene Authentifizierungsmethoden bereit, z.B. EAP-SIM, EAP-TTLS oder EAP-over-LAN (EAPOL). \cite{1561920} In \autoref{fig:eap} wird EAP als Abstraktionsebene dargestellt.\\

\begin{figure}[hbt]
	\centering
	\includegraphics[width=9cm]{figures/EAP}
	\caption{EAP als Abstraktionsebene zur Authentifizierung (angelehnt an: \cite{1561920}).}
	\label{fig:eap}
\end{figure}

EAP wurde zur Unterstützung von Authentifizierung in Netzwerken geschaffen, ohne dass  bei einer neuen Authentifizierung die Infrastruktur angepasst werden muss. Es erlaubt den Einsatz eines Authentication Servers (AAA) und implementiert die Prozesse zwischen Supplicant, Authenticator und Authentication Server. Hierbei können auch mehrere Methoden in Folge genutzt werden und erlaubt so heterogene Netzwerke. Nach Authentifizierungsanfrage (Authentication Request) vom Authenticator an den Supplicant sendet dieser eine Antwort, welche die konkrete Authentifizierung (Benutzer, Password, Zertifikat, etc.) enthält. Der Authenticator erfragt beim Authentication Server die Authentifizierung und schließt den Prozess mit einer Success- oder Failure-Response an den Supplicant ab.

\vspace{.5em}
\subsection{IEEE802.1X}
Der IEEE802.1X (kurz auch dot1x genannt) ist Teil der IEEE802.1 Gruppe von Standards \cite{5409813}. Es dient zur port-based Network Access Control (PNAC) und definiert Authentifizierungsmechanismen für LAN und WLAN. Hierzu setzt es auf das EAP-over-LAN (EAPOL) auf. In \autoref{fig:dot1x} wird die Funktionsweise und die einzelnen Parteien in dot1x dargestellt.\\

\begin{figure}[hbt]
	\centering
	\includegraphics[width=9cm]{figures/dot1x}
	\caption{IEEE802.1X schrittweise in den einzelnen Parteien dargestellt (angelehnt an: \cite{eap}.}
	\label{fig:dot1x}
\end{figure}

Ports werden hierbei in zwei logische Einheiten unterteilt: (1) kontrollierte Ports und (2) unkontrollierte Ports. \cite{aboba2004extensible}\\
Kontrollierte Ports werden durch die dot1x PAE (Port Access Entity) gesteuert und lassen den Netzwerkverkehr für einen Supplicant entweder durch (authorized state) oder blocken diesen ab (unauthorized state).\\
Unkontrollierte Ports werden durch die dot1x PAE zur Übermittlung von EAPOL Paketen verwendet.

\vspace{.5em}
\subsection{RADIUS und Active Directory}
Oft als konkrete Implementierung eingesetzt werden RADIUS (Remote Authentication Dial-In User Service) oder Microsoft Active Directory. Diese bieten zentralisierte AAA-Server und sind dot1x konform, unterstützen also EAPOL. In \autoref{fig:radius} wird der Authentifizierungsprozess in Request und Response dargestellt.

\begin{figure}[hbt]
	\centering
	\includegraphics[width=9cm]{figures/radius}
	\caption{RADIUS Authentifizierung (angelehnt an: \cite{radius}).}
	\label{fig:radius}
\end{figure}

\vspace{.5em}
\subsection{Technische Umsetzung von MAB}
MAB wird meistens als Fallback für EAP / dot1x, also z.B. RADIUS, eingesetzt. Versucht ein Supplicant sich neu zu authentifizieren wird zunächst dessen Netzwerkverkehr geblockt. Der Authenticator versucht nun den Supplicant zu authentifizieren. Hierzu wird zunächst ein EAP-Request-Identity an den Supplicant gesendet, um die Authentifizierung zu initieren. Diesen Request versteht der Supplicant allerdings nicht, da er aufgrund Alters des System oder Niedrigkosten-optimierten Herrstellung EAP nicht implementiert. Diese EAP-Requests werden wiederholt, bis ein dot1x Timeout erfolgt, woraufhin die Authentifizierung abgebrochen wird. \cite{eap}\\

Anschließend beginnt die MAB Authentifizierung. Um die MAC-Adresse des Supplicant zu erlernen, wertet der Authenticator ein einzelnes Paket aus. Hierzu kann fast jedes beliebige OSI-Layer 2 oder 3 Paket genutzt werden. Der Authenticator sendet nun einen Access-Request an den Authentication Server. Dieser Request enthält in den Attributen \emph{Username} und \emph{Password} jeweils die MAC-Adresse des Supplicant. \cite{mab-deployment-guide} Lediglich das Format, in welchem die MAC-Adresse in den Attributen übertragen wird, unterscheidet sich. Zusätzlich findet sich in einigen Systemen noch das Attribute \emph{Calling-Station-ID} wieder, welches ebenfalls mit der MAC-Adresse befüllt und als sechs Gruppen mit jeweils zwei Hexadezimalstellen, in Großbuchstaben und mit Bindestrichen seperiert, übertragen wird. Welches Attribut konkret ausgewertet wird, hängt vom eingesetzten AAA-Server sowie dessen Konfiguration ab.\\

Der Authentication Server nimmt nun eine Bewertung vor und sendet entweder eine Accept- oder Deny-Response an den Authenticator zurück. Wichtige Voraussetzung hierzu ist, dass der Authentication Server die MAC-Adresse des Supplicant kennt, d.h. in einer Datenbank o.ä. zuvor eingepflegt wurde.\\

Der Authenticator erhält bei Erfolg ein Accept zurück und hat nun die MAC-Adresse des Supplicant erlernt. Der Supplicant ist nun auf diesem Port authentifiziert (authorised state) und kann kommunizieren. In \autoref{fig:mab-fallback} wird dieser Prozess noch einmal dargestellt.\\

\begin{figure}[hbt]
	\centering
	\includegraphics[width=9cm]{figures/mab-fallback}
	\caption{MAB als Fallback für RADIUS (Quelle: \cite{mab-deployment-guide}).}
	\label{fig:mab-fallback}
\end{figure}

%

\vspace{1em}
\section{Praxiseinsatz}
In der Praxis wird MAB nicht nur als Fallback für RADIUS genutzt, sondern auch als zusätzliche Sicherheitsebene für \emph{Network Authentication Control (NAC)}. Für Endgeräte, die nicht über Layer 2 hinweg kommunizieren können, wie zum Beispiel IP-Phones, Kameras oder Drucker, die man dennoch nicht ungesichert im Netzwerk angeschlossen lassen möchte, bietet \emph{MAB} eine Möglichkeit diese Endgerät und weitere über verschiedene Mechanismen zu sichern.\\

\vspace{.5em}
\subsection{Heutiger Einsatz}
Heutzutage wird \emph{MAB} vor allem als Fallback genutzt, jedoch wie oben erwähnt, können durch das Provisionieren und Blockieren von MAC-Adressen auf verschiedenen Ebenen Policies erstellt werden, um Endgeräte bereits auf Layer 2 Ebene im Netzwerk zu sperren. Zusätzlich dazu kann durch Architekturen gewährleistet werden, das diese MAC-Adressen durch ausgewählte User gepflegt werden, sodass die Datenbank, in der sie provisioniert werden, nicht durch Dritte veränderbar ist. Dies verhindert zum einen Angriffe wie \emph{MAC-Spoofing}, ermöglicht außerdem es auch mehrere hunderte oder tausende MAC-Adressen an einem zentralen Ort zu pflegen. Hierdurch kann eine Skalierbarkeit in die Breite ermöglicht werden, ohne die Architektur verändern zu müssen oder Endgeräte, die nicht über Layer 2 hinweg kommunizieren können, aus dem Netzwerk auszusperren. Dies hat zusätzlich zur Folge, dass Switches lokal nicht konfiguriert werden müssen, da die Policies und die Datenbank auf dem RADIUS-Server anliegen und die Switches lediglich mit diesem kommunizieren müssen.\\

Diese Konstellation bietet eine weitere Sicherheitsebene, da gewährleistet kann, dass der RADIUS-Server durch ausgewählte Administratoren gepflegt wird und die Switches bekannt sind. Dadurch kann eine Veränderung am Switch erkannt werden, falls diese vorliegt und ebenfalls durch Log-Einträge am RADIUS-Server erkannt werden, falls sich MAC-Adressen, welche am Switch liegen, verändert haben.\\
Allerdings birgt diese Konstellation das Risiko, dass Switches durch Dritte ohne Überwachung ausgetauscht werden können, und eröffnet wiederum den Angriffswinkel des MAC-Spoofings.\\

\begin{figure}[hbt]
	\centering
	\includegraphics[width=9cm]{figures/MAB_heutiger_Einsatz}
	\caption{Heutiger Einsatz für MAB.}
	\label{fig:mab-today}
\end{figure}

\vspace{.5em}
\subsection{Relevanz in Industrie 4.0}
Die Relevanz von \emph{MAB} in der Industrie 4.0 bezieht sich nicht auf die Verfügbarkeit in der Zukunft, sondern darauf wie mit Endgeräten, die nicht über Layer 2 hinweg kommunizieren können, zukünftig umgegangen wird. Dies umfasst u.A. Punkte wie eine hoch flexible Verknüpfung der Datenebene \cite{hirsch2014wandel}. Hier bietet MAB nicht nur die Möglichkeit als Fallback für RADIUS eingesetzt zu werden, sondern insbesondere als Sicherheitsebene zu dienen, welche den großen Vorteil der oben erwähnten flexiblen Verknüpfung bietet.\\

Ein weiterer relevanter Punkt für die Industrie 4.0 wird die Beherrschung komplexer Systeme \cite{botthof2015zukunft}, da zusätzlich zu den in den letzten Jahren angestiegenen Zahlen der IoT-Geräte, auch Altlasten, die durch den Ausbau von Netzen mitgetragen werden, weiterhin sicher angebunden werden müssen. Hierbei kann die erwähnte Maßnahme von Pflege und Provisionierung von MAC-Adressen sehr praktikabel sein, da die Erweiterung von Netzwerken und Technologien keine Kompromitierung darstellen sollte, sondern diese nur erweitern und im besten Falle auch verbessern soll. Im weiteren Sinne kann die Skalierbarkeit, die MAB bietet, auch als weiterer Vorteil für eine flächendeckende Breitbandinfrastruktur \cite{botthof2015zukunft} dargestellt werden. In Sendlers \emph{Industrie 4.0 Grenzenlos} wird auch die verengte Sicht auf die Produktion als Negativbeispiel genannt \cite{sendler2016industrie}, da historisch diese bei industriellen Revolutionen am meisten Aufwand erzeugt und immer wieder optimiert werden muss.\\

Weiterhin nennt er auch, dass in der Industrie 4.0 auf Dienstleister und alle weiteren Aspekte einer Wertschöpfungskette geachtet werden muss, falls diese Revolution erfolgreich sein soll. Wie weiter oben erwähnt, gehören hier auch die Altlasten hinzu, die in wachsenden IoT-Netzwerken immer weniger an Bedeutung finden, jedoch weiter betrachten werden müssen, sei es als Glied der Wertschöpfungskette oder als Dienstleister für andere Produzenten. Auf diesen Punkt geht jedoch die Analyse \emph{Industrie 4.0–Vorgehensmodell} in Bezug auf die Einführung von MAB ein. Hier wird untersucht, ob und wie die bereits bestehenden Produkt- und Prozessebenen angepasst werden müssen, und inwiefern diese die Industrie 4.0 wiederspiegeln \cite{lucia2016industrie}. Diese Analyse, entgegen der bereits erwähnten Punkte, spiegeln eine akkuratere Anpassung einer Wertschöpfungskette an die Industrie 4.0 und können durch konkrete Ist- und Soll-Analysen verglichen, und an die Entwicklung eines Unternehmens sowie von Prozessen angepasst werden.\\

%

\vspace{1em}
\section{Ergebnis}
Zusammenfassend lassen sich die Möglichkeiten und sicherheitstechnischen Elemente von MAB in verschiedenen Aspekten aufzeigen. Ein Faktor ist es Endgeräte, die nicht über Layer 2 hinweg kommunizieren können, in eine heterogene Netzwerklandschaft miteinbeziehen, ohne größere Sicherheitsrisiken einzugehen. Dies ist vor allem durch zentrales Provisionieren, Pflegen von Datenbanken und dem konkreten Ausschluss von unbekannten MAC-Adressen möglich. Dies verursacht zwar einen gewissen Aufwand auf Seiten von Netzwerkadministratoren und Haltern von Endgeräten, da diese zuerst aufgenommen und gepflegt werden müssen. Jedoch kann durch den Aufbau von Prozessen und Policies, die diese Endgeräte durchlaufen, gewährleistet werden, dass neue Endgeräte ebenfalls schneller und sicherer ans Netz gebracht werden können. Durch Architekturen und RADIUS-Server, wie z.B. die \emph{Cisco ISE}, lässt sich dies ebenfalls, in einer flexiblen Verknüpfung für breite Netzwerklandschaften erfüllen.\\

Abgesehen davon MAB als Sicherheitsebene zu nutzen, lässt sich die Funktion auch als Fallback-Ebene für Endgeräte nutzen, die sich bereits im Netzwerk befinden. Dies kann nützlich sein falls bei EAP-basierten Verbindungen Fehler auftreten oder in Netzwerken mehrere Ebenen und Verbindungsmöglichkeiten aufgesetzt werden. Hierbei kann MAB als Fallback-Ebene für jegliche Endgeräte genutzt werden, da diese ebenfalls eine MAC-Adresse im Netzwerk haben, welche in Datenbanken verwaltet und provisioniert werden können.\\

%

\vspace{.5em}
\subsection{Vor- und Nachteile}

An diesem Punkt lassen sich auch die Vor- und Nachteile von MAB aufzählen. Wie eben erwähnt lässt sich die Sicherheitsebene relativ einfach durch Abfragen von MAC-Adressen realisieren. Im aufgeführten Beispiel wurde die Architektur von Cisco erwähnt auf der diese Authentifizierunsmethode basiert. Hierbei werden Endgeräte nach ihren MAC-Adressen sortiert sowie Policies auf Basis dieser Gruppen erstellt. Am Ende jeder Policy steht ein sogenannter \emph{Deny-All}, welcher zur Folge hat, das jegliche Endgeräte, die zu keiner dieser Gruppen passen, von dem RADIUS-Server nicht authentifiziert werden. Dabei ist zu erwähnen, dass eine zertifikatsbasierte Authentifierung nicht notwendig ist. Somit ist die Pflege von Zertifikaten, obgleich es sich um eine intern- oder extern-basierte Zertifikatsinfrastruktur handelt, von vornherein ausgeschlossen.\\

Zusätzlich dazu benötigt der Switch nur ein einzelnes Paket, um die MAC-Adresse des Endgeräts zu erlernen, welches sich authentifizieren möchte. Damit ist die Netzwerklast geringer als durch zertifikats- oder Username-Passwort-basierte Authentifizierungsmethoden. Dies basiert alles darauf, dass nicht über Layer 2 hinweg kommuniziert werden muss, wodurch keine TCP-IP Verbindung aufgebaut werden muss und rein auf der Data-Link-Layer kommuniziert werden kann (siehe \autoref{fig:mab}).\\

\begin{figure}[hbt]
	\centering
	\includegraphics[width=9cm]{figures/MAB_Layer_2.jpg}
	\caption{Grenze von Endgeräte- und User-basierter Authentifizerung.}
	\label{fig:mab}
\end{figure}

Jedoch ist hier auch zu erwähnen, dass gewisse Nachteile bei Nutzung von MAB entstehen. Zunächst müssen Datenbanken gepflegt werden. Dies kann bei relativ kleinen Unternehmen übersichtlich erscheinen, jedoch bei zunehmender Unternehmensgröße unübersichtlich und komplex werden. Anbindung und Pflege von mehreren tausend Endgeräten, vor allem wenn diese erst neu dokumentiert werden müssen, kann schnell zu einem Mehraufwand werden, welcher sich im Hinblick auf Kosten-Leistungs-Analysen nicht zu lohnen scheinen mag. Hier muss Zweck gegen Risiko abgewogen werden, ob diese Geräte überhaupt noch am Netz angebunden bleiben sollen.\\

Weiterhin, ist keine personenbezogene Authentifizierung möglich. Da weder zertifikats- noch Username-Passwort-basierte Authentifizierung auf Layer 2 möglich ist, kann nur das Endgerät authentifiziert werden, nicht der Nutzer des Endgeräts. Dies erschwert die Einbindung von Identity- und Access-Management in dieser Form von Authentifizierung. Zusätzlich dazu müssen die Switches, an denen MAB durchgeführt werden soll, konfiguriert werden, sodass sie mit dem RADIUS-Server kommunizieren und durch diesen auch authentifiziert werden können. Dies kann je nach Anzahl der zu konfigurierenden Switches einen erheblichen Mehraufwand erzeugen, vor allem wenn diese händisch aufgespielt werden (siehe \autoref{fig:mab_configuration}).\\

\begin{figure}[hbt]
	\centering
	\includegraphics[width=9cm]{figures/Server_Switch.jpg}
	\caption{Architektur Skizze einer MAB-Konfiguration.}
	\label{fig:mab_configuration}
\end{figure}

Letztlich ist noch der relativ einfache Angriffswinkel des Spoofings aufzunehmen, denn falls die Switches offenliegen und durch Dritte austauschbar sind, kann dies zum Vorteil eines Angreifers ausgenutzt werden, um Endgeräte im Netz zu authentifizieren und so eine Backdoor zu öffnen. Dies kann durch physikalisches Absichern der Switches in Schutzschränken erreicht werden oder durch Anbringen der Switches an Orten, an die Angreifer ohne Weiteres nicht herankommen, wie zum Beispiel an hohen Decken. Hierbei können Angreifer entweder den Switch selbst austauschen, um diesen als Angriffsvektor zu nutzen, oder an einem bestehenden Switch ein Endgerät, bei dem die MAC-Adresse bekannt ist, diese vortäuschen, um so Informationen über das Netzwerk zu gewinnen (siehe \autoref{fig:mac_spoofing}).\\

\begin{figure}[hbt]
	\centering
	\includegraphics[width=9cm]{figures/Angriffsszenario_Spoofing.jpg}
	\caption{Angriffsszenario MAC-Spoofing}
	\label{fig:mac_spoofing}
\end{figure}

\vspace{.5em}
\subsection{Zukunftsausblick}

Zuletzt sollte noch darauf eingegangen werden, wie zukunftsträchtig sich MAB darstellt und in wie weit sich diese Architektur für die Industrie 4.0 eignet.\\

Wie zuvor dargestellt werden in heutigen Analysen in Bezug auf die Industrie 4.0 hauptsächlich neue Infrastrukturen, IoT-Geräte oder komplexe Systeme betrachtet. Jedoch wird selten in diesen Analysen auf die Altlasten in Architekturen und Netzwerken eingegangen. In Ist- und Soll-Analysen muss jedoch zwingend auf diese eingegangen werden, damit im Ausbau und in der Weiterentwicklung durch diese Altlasten keine Komplikationen auftreten, welche man im Nachhinein auflösen muss. Von daher hat MAB aus diesem Blickwinkel einen Nutzen für die Industrie 4.0. Jedoch müssen die Vor- und Nachteile zwischen Sicherheit und Aufwand abgewogen werden.\\

In zukunftsorientierten Architekturen und Netzwerken, die sich immer weiter in Richtung der Industrie 4.0 bewegen, ist es nicht vernachlässigbar auch die Altlasten, welche sich im System bereits befinden, weiterhin sicher anzubinden. Hierbei bietet MAB eine gute und relativ einfach aufzusetzende Lösung. Jedoch muss hierbei, je nach Größe der Netzwerke und der anzubindenden Endgeräte, beachtet werden, dass dies einen Mehraufwand erzeugen kann, und gegenüber neu anzuschaffenden Gerät abgewogen werden muss. In der Regel ist es sinnvoller durch MAB diese Endgeräte anzubinden und per RADIUS-Server zu authentifizieren, da dies ebenfalls eine Möglichkeit bietet, einen Überblick aller vorhandenen Endgeräte zu schaffen sowie gleichzeitig ungenutzte Geräte auszuschließen oder sogar zu ersetzen. In der Regel heißt dies auch, dass hierdurch ein Fallback geschaffen werden kann, um Geräte bei DNS, PKI oder AD Problemen durch MAB im Netz zu halten, um so z.B. Produktivität aufrechtzuerhalten als auch Endgeräte wie Kameras am laufen zu lassen, ohne diese weiter ins Netz einbinden zu müssen.\\

Als Fazit lässt sich also sagen, dass MAB eine gute Möglichkeit bietet ein Sicherheitsniveau zu schaffen, und gleichzeitig eine Fallback-Lösung einzurichten. Hierbei muss jedoch der Mehraufwand aufgebracht werden, um dieses Niveau zu erreichen. Die weitere Pflege ist allerdings im Vergleich zu herkömmlicher Konfiguration einzelner Endgeräte etwas einfacher aufrechtzuerhalten.

%

\vspace{1em}
\bibliographystyle{IEEEtran}
\bibliography{references}

\end{document}