\documentclass[12pt,oneside,a4paper]{article}
\usepackage[utf8]{inputenc}
\usepackage[english,ngerman]{babel}
\usepackage{graphicx}
\usepackage{fancyhdr}

%\pagestyle{fancy}


%\fancyhf{}
%\lhead{\textbf{Hochschule Wismar} \\
%			University of Applied Sciences \\
%			Technology, Business and Design \\
%			Fakult\"at f\"ur Ingenieurwissenschaften, Bereich EuI \\}
%\rhead{\includegraphics[height=6\baselineskip,right]{bilder/HS-Wismar_Logo-FIW_2010-01.jpg}}
%\setlength{\headheight}{2.5cm}

\begin{document}
\title{Mac Authentication Bypass}
\begin{tiny}
\author{David Schunke, Umut-Vural Mitiler \\
\\
Hochschule Wismar \\
University of Applied Sciences, Technology
}
\end{tiny}
\date{} 

\maketitle 
%Mac Authentication Bypass
%Industrie 4.0
%Heterogene Landschaft
%Skalierbares Netzwerk


\begin{center}
\textbf{Abstract}\\
\end{center}
\\


\section{MAB Übersicht}
Hier kommt alles bezüglich Einleitung MAB

%MAB Cisco Link
%Layer 2 Security Link?
%RADIUS vllt

\section{MAB heutiger Einsatz}
Hier kommt alles bezüglich Einsatz in der heutigen Zeit
%Fallback
%Drucker, Telefone, IP-Phones, Kameras


\section{Relevanz von MAB in Industrie 4.0}
Hier kommt alles zur Relevanz von Industrie 4.0
%Was ist Industrie 4.0
%Skalierbarkeit, Heterogene Landschaft, Einbindung von "Leichen"

\section{Vor- und Nachteile von MAB}
Hier zählen wir alle vor und Nachteile von MAB auf und kommen danach zum Fazit
Hier würde ich einfach 1 zu 1 die Powerpoint runterschreiben, das haben wir bereits sehr ausführlich gemacht
%Powerpoint verweisen
%Vorteile Nachteile
%Skalierbarkeit vs L2 Security (Spoofing)

\section{Fazit}
Hier kurz was MAB in der Industrie 4.0 für Netzwerke bedeutet und wie es nicht nur als Fallback genutzt werden kann um Netzwerke zu härten.
%Fazit
%Zukunft und Vergangenheit







\begin{thebibliography}{unsrt}

\end{thebibliography}
 


\end{document}